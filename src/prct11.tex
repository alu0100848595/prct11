\documentclass{beamer}
\usepackage[utf8]{inputenc}
\usepackage{graphicx}

\newtheorem{definicion}{Definición}
\newtheorem{ejemplo}{Ejemplo}

%%%%%%%%%%%%%%%%%%%%%%%%%%%%%%%%%%%%%%%%%%%%%%%%%%%%%%%%%%%%%%%%%%%%%%%%%%%%%%%
\title[Presentación con Beamer]{Número Pi}
\author[Oscar Andrés Díaz Sánchez]{Oscar Andrés Díaz Sánchez}
\date[25-03-2014]{25 de marzo de 2014}
%%%%%%%%%%%%%%%%%%%%%%%%%%%%%%%%%%%%%%%%%%%%%%%%%%%%%%%%%%%%%%%%%%%%%%%%%%%%%%%

\usetheme{Madrid}
%\usetheme{Antibes}
%\usetheme{tree}
%\usetheme{classic}

%%%%%%%%%%%%%%%%%%%%%%%%%%%%%%%%%%%%%%%%%%%%%%%%%%%%%%%%%%%%%%%%%%%%%%%%%%%%%%%
\begin{document}
  
%++++++++++++++++++++++++++++++++++++++++++++++++++++++++++++++++++++++++++++++  
\begin{frame}

  %\includegraphics[width=0.15\textwidth]{img/ullesc.png}
  \hspace*{7.0cm}
  %\includegraphics[width=0.16\textwidth]{img/fmatesc.png}
  \titlepage

  \begin{small}
    \begin{center}
     Facultad de Matemáticas \\
     Universidad de La Laguna
    \end{center}
  \end{small}

\end{frame}
%++++++++++++++++++++++++++++++++++++++++++++++++++++++++++++++++++++++++++++++  

%++++++++++++++++++++++++++++++++++++++++++++++++++++++++++++++++++++++++++++++  
\begin{frame}
  \frametitle{Índice}  
  \tableofcontents[pausesections]
\end{frame}
%++++++++++++++++++++++++++++++++++++++++++++++++++++++++++++++++++++++++++++++  


\section{Primera Sección}


%++++++++++++++++++++++++++++++++++++++++++++++++++++++++++++++++++++++++++++++  
\begin{frame}

\frametitle{Primera Sección}

\begin{definicion}
\alert{El número pi}~\cite{wiki} es la relación entre la longitud de una circunferencia y su diámetro, en geometría euclidiana. Es un número irracional y una de las constantes matemáticas más importantes. Se emplea frecuentemente en matemáticas, física e ingeniería. El valor numérico de pi, truncado a sus primeras cifras, es el siguiente:
\begin{center}
   $\pi \approx 3,14159265358979323846 \; \dots $
\end{center}
\end{definicion}

\end{frame}
%++++++++++++++++++++++++++++++++++++++++++++++++++++++++++++++++++++++++++++++  

\section{Segunda Sección}

%++++++++++++++++++++++++++++++++++++++++++++++++++++++++++++++++++++++++++++++  
\begin{frame}

\frametitle{Segunda Sección}

\begin{block}{Ejemplo}
  \begin{itemize}
  \item
    $S = \pi r^2 \simeq \left ( \frac{8}{9} \cdot d \right )^2 = \frac{64}{81} d^2 = \frac{64}{81} \left(4 r^2\right) $
  \pause

  \item
    $\pi \simeq \frac{256}{81} = 3{,}16049 \ldots $
  \pause

  \item
    $\pi \approx 3 + \frac{1}{8} = 3,125 $
  \pause
  
  \item
    $
    \arcsin {x} = x + \frac{1}{2} \cdot \frac {x^3}{3} + \frac{1 \cdot 3}{2\cdot 4} \cdot \frac {x^5}{5} + \frac{1\cdot 3\cdot 5}{2\cdot 4\cdot 6} \cdot \frac{x^7}{7} + \ldots $

  \end{itemize}
\end{block}

\end{frame}
%++++++++++++++++++++++++++++++++++++++++++++++++++++++++++++++++++++++++++++++  

\section{Ejercicios}

\subsection{Una subsección}
%++++++++++++++++++++++++++++++++++++++++++++++++++++++++++++++++++++++++++++++  
\begin{frame}
\frametitle{Título de la diapositiva}

Texto de la diapositiva
\end{frame}
%++++++++++++++++++++++++++++++++++++++++++++++++++++++++++++++++++++++++++++++  

\subsection{Creación de diapositivas}

%++++++++++++++++++++++++++++++++++++++++++++++++++++++++++++++++++++++++++++++  
\begin{frame}
\frametitle{Diapositivas}

\begin{definition}
  Un ejemplo de definición
\end{definition}

\begin{example}
  \begin{itemize}
    \item <1-> Practica\pause
    \item <2-> de \pause
    \item <3-> Beamer \pause
    \item <4-> de \alert{Tecnicas Experimentales}~\cite{guia}   
  \end{itemize}
\end{example}

\end{frame}
%++++++++++++++++++++++++++++++++++++++++++++++++++++++++++++++++++++++++++++++  

\subsection{Otra subseccion}
%++++++++++++++++++++++++++++++++++++++++++++++++++++++++++++++++++++++++++++++  
\begin{frame}
\frametitle{Este es otro Título}

\begin{definicion}
  Otra definición 
\end{definicion}

\begin{ejemplo}
  \begin{enumerate}
    \item
      Primero
      \pause

    \item
      Segundo 

  \end{enumerate}
\end{ejemplo}

\end{frame}
%++++++++++++++++++++++++++++++++++++++++++++++++++++++++++++++++++++++++++++++  

\section{Bibliografía}
%++++++++++++++++++++++++++++++++++++++++++++++++++++++++++++++++++++++++++++++  
\begin{frame}
  \frametitle{Bibliografía}

  \begin{thebibliography}{10}

    \beamertemplatebookbibitems
    \bibitem[Wikipedia]{wiki}  
    El número pi 
    {\small $http://es.wikipedia.org/wiki/Numero$ $\_$ $pi$}

    \beamertemplatebookbibitems
    \bibitem[Campus Virtual, 2014]{guia}  
    Campus Virtual. 
    (2014) 
    {\small $http://campusvirtual.ull.es/1314/pluginfile.php/197727/mod$ $\_$ $resource/content/2/p11.pdf$}

  \end{thebibliography}
\end{frame}

%++++++++++++++++++++++++++++++++++++++++++++++++++++++++++++++++++++++++++++++  
\end{document}
